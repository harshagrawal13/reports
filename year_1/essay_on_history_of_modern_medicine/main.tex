\documentclass{article}
\usepackage[font=small]{caption}
\usepackage[hang,flushmargin]{footmisc}
\usepackage{hyperref}
\usepackage[margin=0.9in]{geometry}
\usepackage{natbib}
\bibliographystyle{apalike}
\usepackage{csquotes}
\captionsetup{justification=centering}
\usepackage{url}

\title{Compulsory Vaccination in 19\textsuperscript{th} Century Britain}
\author{Harsh Agrawal \\ \href{ha1822@ic.ac.uk}{ha1822@ic.ac.uk} \\ Molecular Bioengineering \\ Imperial College London}
\date{\today}

\begin{document}
\maketitle

The rapid spread of Smallpox in the late 18\textsuperscript{th} century caused
havoc in Britain, killing millions in its wake. With the discovery of the
vaccine in 1796, Edward Jenner bought new hope against this vile predator. To
ensure widespread protection, the government passed a series of acts from 1840
to 1853 \citep{fenner_1988}, making the vaccine free and compulsory for all
children under the age of 14. While many welcomed the vaccine as a breakthrough
in public health, others vehemently opposed it, arguing that compulsory
vaccination infringed on their personal liberties and religious beliefs. This
essay aims to explore both sides of the debate in themes such as therapeutic
efficacy, overall implementation, and the state of misinformation; it further
aims to discuss, in retrospect, the overall effectiveness of mandatory
vaccination.

It is crucial to first understand the need for vaccination in
19\textsuperscript{th} century Britain. Smallpox was a malignant killer that
had no available cure \citep{millward_2019}. The only therapeutic procedure
used was \textit{inoculation} --- subcutaneous instillation of smallpox virus
taken from a fresh matter of an affected individual into a nonimmune
individual. Apart from the uncertainty to obtain immunity, this process was
dangerous and could also spread other vile diseases such as \textit{syphilis}.
Thus, when Jenner discovered the vaccine, it was considered a massive
breakthrough. He heavily vouched for vaccination around Britain and the world
and presented various suppositions in his book \textit{An Inquiry into the
    Causes and Effects of the Variolae Vaccinae} published in 1798
\citep{jenner_2012}. His claims included that the vaccine possessed such
``singular properties as indisputably to render it one of the most important
discoveries that have ever been made for the benefit of mankind.\textquote''

Some of these claims were credibly proven through extensive testing conducted
in 1801\citep{WHO}; other claims such as that of the vaccine rendering
permanent immunity, although unproven, greatly reinforced the dire need for
mandatory vaccination against Smallpox in early 19\textsuperscript{th} century.

Official statistics on deaths due to smallpox in London reported before and
after the introduction of voluntary vaccination also showed promising results
\citep{edwardes_2007}. Deaths reported due to smallpox in the three decades
(1771--1801) before the introduction of the vaccine stood between the range of
~18,000 to\~20,000 mortalities per decade. The death rate fell dramatically in
the decades that followed the introduction of the vaccine (1801--1830). Only
12,534 deaths were reported from 1801--1811, followed by an even sharper
decline with only 7,858 reported deaths from the period of
1811--1820\footnote{The writer does not explicitly claim that the reduction in
    mortality was caused due to vaccination.}. Around the mid-1840s when the
promotion of vaccination was in full gear, the reports mark the presence of
\textit{no deaths} from the period of 1843--1846, fortifying pro-vaccinators'
beliefs on the miraculous nature of the vaccine\citep{edwardes_2007_}

However persuasive, by the end of the century, there were many who rigorously
questioned the efficacy of the vaccine. Dr.~Alfred Russel Wallace, one of the
most prestigious scientists (known to have independently discovered Darwin's
theory of evolution) challenged the notion of mandatory vaccination and several
assertions regarding vaccine efficacy in his book\citep{wallace_wheeler_1889}.
He refuted that death from smallpox was lower for vaccinated populations, the
attack rate was lower for the vaccinated populations, and that vaccination
alleviated the clinical symptoms of smallpox. He used actuarial statistics to
support his arguments, such as the analysis of life tables and mortalities.
While his opponents argued that the peaks in mortality, such as that of
1870--1873, were due to a widespread anti-vaccination movement, Wallace
concluded that the same peaks were due to increased vaccination rates at the
time when cumulative penalties were introduced and fewer dared to challenge the
vaccination law \citep{CDC2009}.

He argued that the problem of determining vaccination status was extremely
difficult. Due to the incompleteness of the epidemiologic data for vaccination
status, the vaccination status of 30\%–70\% of smallpox deaths was unknown.
Moreover, if a person contracted the disease shortly after a vaccination, it
was often entirely unclear if the patient was to be categorized as vaccinated
or unvaccinated. Finally, Wallace also believed that doctors were more willing
to report a death from smallpox in an unvaccinated patient and that this led to
a serious bias and an overestimation of vaccine efficiency.

Apart from Alfred Wallace's claims, evidence from 1880 onwards also confirmed
that Jenner's contention about the lifelong immunity of the vaccine was
incorrect\citep{colin_r}. Protection induced by primary vaccination in
childhood did not necessarily last into adulthood. In Sheffield Borough
Hospital during the 1887–88 outbreak, none of the nurses and attendants who had
been re-vaccinated got smallpox, whereas 6 cases ensued in those who had
received only a single dose as a child\footnote{The information was taken
    directly from the secondary source. The primary source wasn't found.}. It
proved that while the vaccine was beneficial, one needed to be vaccinated again
for prolonged immunity.

Apart from the inferred vaccine efficacy and its ability to curb contagion, it
can also be argued whether it helped curb the downright deteriorating effects
of false information and lies spread by the Anti-vaccinators. John Campbell, a
medical officer of health for Gloucester City and Port, in 1897 reported the
anti-vaccinators had caused it to be rumored far and near that the true cause
of the epidemic was the \textit{unhealthy condition} of the city and not the
want of vaccination\citep{faherty}.

Even more so, outspoken people such as physician Benjamin Moseley (1742--1819)
alarmed readers with luridly worded arguments against the abominable practice
of introducing a ``bestial humor into the human frame'' while hinting darkly at
the ``strange mutations from quadruped sympathy'' that might result in, as well
as relating fantastical accounts of vaccinated children sprouting cow hair or
developing facial features distorted ``to resemble that of an
Ox''\citep{moseley_1807}\citep{the_morgan_library}. Misinformation related to
the vaccine grew with the widespread anti-vaccination sentiment. As a result,
the vaccination rates in some parts of Britain drastically fell by the end of
the century. A report by the Vaccination Inquirer in 1896 described the
horrific reality. In cities with the least vaccination in the country, such as
Gloucester, almost \textit{83\%} of the population was failing to comply with
the law\citep{faherty}.

These lies were often spread via rhetoric and artistic means allowing for an
easier diffusion into the socio-economic fabric of 19\textsuperscript{th}
Britain. Notable examples include the \textit{Cow Puck} \citep{gillray_1802},
depicting a rather menacing Edward Jenner vaccinating patients who develop
features of cows, and \textit{The Vaccination Monster}
illustration\citep{williams_1802}, depicting vaccinators feeding their babies
to a large, brooding vaccination monster.

While some of these artistic depictions spread lies and misinformation, they
also did convey justifiable distrust, fear, anger, and in concavity, the true
sentiments of the population regarding mandatory vaccination. It also presented
the deep-rooted fact that the implementation of mandatory vaccination was
deeply flawed by being unfair, thoroughly class-based, and coercive; the poor,
working class were subjected to the full force of the law while better-off were
provided with safer vaccines and could easily avoid punishment if they did not
comply \citep{faherty}.

Many, including Alfred Wallace, righteously argued that the mandatory nature of
vaccination was an infringement of personal
choice\citep{the_national_archives_2022}. William Tebb, President of the London
Society for the Abolition of Compulsory Vaccination, 1887, also presented that
the state has no right to encroach upon \textit{parental responsibility} or to
impose either religious or medical dogmas upon the people of this country on
any pretense whatever\citep{tebb_1887}. Others believed that their child
shouldn't have to go through a painful and dangerous procedure and that there
were potential risks involved that need to be addressed. This claim was further
supported after it was found that the widespread arm-to-arm vaccination, used
until 1898, did in fact carry substantial risks, and the instruments used could
contribute to severe adverse reactions\citep{baxby_2002}.

Overall, the arguments and examples presented above demonstrate the ambiguity
in determining the effectiveness of mandatory vaccination in
19\textsuperscript{th} century Britain. While Edward Jenner's initial
experiments proving the therapeutic benefits of his engineered vaccine ---
which he considered to have possessed singular properties to prevent smallpox
--- were perfectly valid, the vaccine had failed to prevent the smallpox
epidemics of 1857--59, 1863--65, and 1870--72 \citep{colin_r}. Similarly,
although Alfred Wallace's claims were quite reasonable, inferential statistics
would have proven to be more helpful in proving his claims, which did not yet
exist. The statistical approach could simply not resolve the issue of vaccine
efficiency; thus, each side was free to choose the interpretation that suited
its needs best. Although, it's safe to say that, in retrospect, the historical
dance between mandatory vaccination and its opposition paved the way for the
generation of more robust and rigorous methodologies, such as population-based
risk assessments, regarding vaccine safety in the entire scientific community.
Both sides were crucial in reducing smallpox mortality in
19\textsuperscript{th} Britain and eventually eradicating the disease for the
better.

\bibliography{main}

\end{document}